\chapter{Mensajes}
\label{appendix:Mensajes}

%====================EJEMPLO=====================================
\subsection{MSJ0.1 Activación de cuenta }
\cdtLabel{MSJ0.1}{}

  Correo electrónico enviado a los aspirantes que han solicitado su generación de cuenta en el \cdtRef{Actor:SAEV2.0}{SAEV2.0}, el cual contiene la liga de activación de su cuenta para realizar su registro de información de ingreso a la licenciatura en derecho de la Escuela Libre de Derecho

  \begin{tabular}{ m{.09\textwidth} m{.78\textwidth}  }%
    \cellcolor{gray3} De: &  \underline{{ \it Correo electrónico oficial del SAEV2.0 }} \\
    \cellcolor{gray3} Para: & \underline{{ \it Correo electrónico aspirante }} \\
    \cellcolor{gray3} Asunto: & Activación de cuenta SAEV2.0-ELD \\
    \multicolumn{2}{ c }{\cellcolor{gray3} Texto: }\\ \\
  \end{tabular}

  \noindent Estimado aspirante:

  \noindent Para que tu cuenta de usuario sea activada, debes dar clic en el siguiente link:

    \begin{center}
      \underline{{\it Link}}
    \end{center}

  Cuentas con 48 horas para realizar la activación de tu cuenta, de lo contrario, deberás de solicitar un nuevo link de activación ...

\noindent \underline{{\it Escuela Libre de Derecho}}\\
\noindent \underline{{\it Datos de la ELD}}

\subsection{MSJ0.2 Operacion Exitosa }
\cdtLabel{MSJ0.2}{}

Mensaje en el sistema que indica que la operación se ha realizado exitosamente.

  \noindent Operación Exitosa


%====================CELULA1=====================================
%Introducir los correspondientes a la GESTIÓN DE EMPLEADOS



%====================CELULA2=====================================
%Introducir los correspondientes a la GESTIÓN DE INVENTARIO



%====================CELULA3=====================================
%Introducir los correspondientes a la GESTIÓN DE PRÉSTAMOS



%====================CELULA4=====================================
%Introducir los correspondientes a la GESTIÓN DE USUARIOS



%====================CELULA5=====================================
%Introducir los correspondientes a la GESTIÓN DE CREDENCIALES



